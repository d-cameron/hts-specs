\documentclass[10pt]{article}
\usepackage{color}
\definecolor{gray}{rgb}{0.7,0.7,0.7}
\usepackage{framed}
\usepackage{enumitem}
\usepackage{longtable}

\addtolength{\textwidth}{3.4cm}
\addtolength{\hoffset}{-1.7cm}
\addtolength{\textheight}{4cm}
\addtolength{\voffset}{-2cm}

\begin{document}

\begin{table}[h]
{\small
\begin{tabular}{|l|l|l|l|l|l|r|}
  \cline{1-7}
  \multicolumn{4}{|c|}{\bf Field} & \multicolumn{1}{c|}{\bf Description} & \multicolumn{1}{c|}{\bf Type} & \multicolumn{1}{c|}{\bf Value} \\\cline{1-7}
  \multicolumn{4}{|l|}{\sf magic} & Magic string & {\tt char[4]} & {\tt CSI\char92 1}\\\cline{1-7}
  \multicolumn{4}{|l|}{\sf min\_shift} & \# bits for the minimal interval & {\tt int32\_t} & [14]\\\cline{1-7}
  \multicolumn{4}{|l|}{\sf depth} & Depth of the binning index & {\tt int32\_t} & [5]\\\cline{1-7}
  \multicolumn{4}{|l|}{\sf l\_aux} & Length of auxiliary data & {\tt int32\_t} & [0]\\\cline{1-7}
  \multicolumn{4}{|l|}{\sf aux} & Auxiliary data & {\tt uint8\_t[l\_aux]} & \\\cline{1-7}
  \multicolumn{4}{|l|}{\sf n\_ref} & \# reference sequences & {\tt int32\_t} & \\\cline{1-7}
  \multicolumn{7}{|c|}{\textcolor{gray}{\it List of indices (n=n\_ref)}} \\\cline{2-7}
  & \multicolumn{3}{l|}{\sf n\_bin} & \# distinct bins & {\tt int32\_t} & \\\cline{2-7}
  & \multicolumn{6}{c|}{\textcolor{gray}{\it List of distinct bins (n=n\_bin)}} \\\cline{3-7}
  & & \multicolumn{2}{l|}{\sf bin} & Distinct bin & {\tt uint32\_t} & \\\cline{3-7}
  & & \multicolumn{2}{l|}{\sf loffset} & (Virtual) file offset of the first overlapping record & {\tt uint64\_t} & \\\cline{3-7}
  & & \multicolumn{2}{l|}{\sf n\_chunk} & \# chunks & {\tt int32\_t} & \\\cline{3-7}
  & & \multicolumn{5}{c|}{\textcolor{gray}{\it List of chunks (n=n\_chunk)}} \\\cline{4-7}
  & & & {\sf chunk\_beg} & (Virtual) file offset of the start of the chunk & {\tt uint64\_t} & \\\cline{4-7}
  & & & {\sf chunk\_end} & (Virtual) file offset of the end of the chunk & {\tt uint64\_t} & \\\cline{1-7}
  \multicolumn{4}{|l|}{{\sf n\_no\_coor} (optional)} & \# unmapped unplaced reads ({\sf RNAME} *) & {\tt uint64\_t} & \\\cline{1-7}
\end{tabular}}
\end{table}

\noindent
The following functions generalise those given in the SAM specification for a BAI-style binning scheme.
Note that in CSI \textit{depth} refers to the scheme's maximal depth, i.e., the level number of the scheme's smallest bins, and the single-bin level spanning the entire coordinate range is level 0.
Hence the BAI-style binning scheme, with six levels in total, is represented by $\mbox{\sf min\_shift} = 14, \mbox{\sf depth} = 5$.

CSI index files may contain metadata pseudo-bins for each reference sequence, with the same contents as BAI pseudo-bins.
In CSI, the pseudo-bins have bin number $\mbox{\sf bin\_limit}(\mbox{\sf min\_shift}, \mbox{\sf depth}) + 1$.

{\footnotesize
\begin{verbatim}
/* calculate bin given an alignment covering [beg,end) (zero-based, half-close-half-open) */
int reg2bin(int64_t beg, int64_t end, int min_shift, int depth)
{
    int l, s = min_shift, t = ((1<<depth*3) - 1) / 7;
    for (--end, l = depth; l > 0; --l, s += 3, t -= 1<<l*3)
        if (beg>>s == end>>s) return t + (beg>>s);
    return 0;
}

/* calculate the list of bins that may overlap with region [beg,end) (zero-based) */
int reg2bins(int64_t beg, int64_t end, int min_shift, int depth, int *bins)
{
    int l, t, n, s = min_shift + depth*3;
    for (--end, l = n = t = 0; l <= depth; s -= 3, t += 1<<l*3, ++l) {
        int b = t + (beg>>s), e = t + (end>>s), i;
        for (i = b; i <= e; ++i) bins[n++] = i;
    }
    return n;
}

/* calculate maximum bin number -- valid bin numbers range within [0,bin_limit) */
int bin_limit(int min_shift, int depth)
{
    return ((1 << (depth+1)*3) - 1) / 7;
}
\end{verbatim}
}

\end{document}
